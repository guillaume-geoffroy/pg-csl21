



Partial metric spaces were  introduced in the early nineties as a variant of metric spaces in which self-distances can be non-zero. Such spaces have attracted much attention in program semantics
\cite{bkmp:partial-metrics, Bukatin1997, doi:10.1111/j.1749-6632.1994.tb44144.x, Schellekens2004, Samet:2013aa, Stubbe2018, HE201999}, due to their compatibility with standard constructions from both domain theory (since their topology is $T_{0}$) and usual metric topology (\textit{e.g.} Cauchy sequences, completeness, Banach-fixed point theorem) \cite{bkmp:partial-metrics, doi:10.1111/j.1749-6632.1994.tb44144.x}.
\emph{Generalized partial metric spaces}, \textit{i.e.} partial metric spaces whose metric takes values over an arbitrary quantale \cite{Hofmann2014}, are well-investigated too \cite{AGT2000,AGT7849}. 

In this paper we will only be concerned with partial metrics taking values over a \emph{commutative integral} quantale \cite{Hofmann2014}, of which we recall the definition below.
% (which is the only kind of quantale we will consider here): %, as we will use these concepts throughout this paper.

\begin{definition} A \emph{commutative integral quantale} is a triple $(Q, \quantaleop, \quantaleleq)$ where:
\begin{itemize}
\item $(Q, \quantaleleq)$ is a complete lattice,
\item $(Q, \quantaleop)$ is a commutative monoid,
\item $\quantaleop$ commutes with arbitrary $\sup$s,
\item the largest element of $Q$ is neutral for $\quantaleop$.
\end{itemize}
\end{definition}

For example, $([0,\infty], +, \geq)$ is a commutative integral quantale whose largest element is $0$ (notice the reversed ordering). It is straightforward to check that for all commutative integral quantales $Q,R$, the product monoid $Q \times R$ equipped with the product ordering is also a commutative integral quantale. In addition, for all posets $X$, the set of monotonically increasing (respectively, decreasing) functions from $X$ to $Q$, equipped with the pointwise monoid operation and the pointwise ordering, is also a commutative integral quantale. Another example of commutative integral quantale is given by the lattice of ideals of any commutative ring, with the product of ideals as the monoid operation.


We recall now the definition of a generalized partial metric space:


\begin{definition} A \emph{generalized partial metric space} (in short, GPMS) is the data of a set $X$, a commutative integral quantale $Q$ and a function $d : X \times X \to Q$ such that: \begin{itemize}
\item for all $x,y \in X$, $d(x,x) \quantalegeq d(x,y)$,
\item for all $x,y \in X$, if $d(x,x) = d(x,y) = d(y,y)$, then $x = y$,
\item for all $x,y \in X$, $d(x,y) = d(y,x)$,
\item for all $x,y,z \in X$, $d(x,z) \quantaleop d(y,y) \quantalegeq d(x,y) \quantaleop d(y,z)$.
\end{itemize}
\end{definition}

For every metric space $(X,d)$, the structure $(X, ([0,\infty], +, \geq), d)$ is a GPMS: notice that the triangular inequality in this definition is the opposite of the usual one, which corresponds to the fact that the quantale ordering on $\mathbb{R}$ is the dual of the usual ordering. 
As is well-known \cite{bkmp:partial-metrics}, any real-valued GPMS $(X,[0,\infty],d)$ induces a metric $d^{*}$ by letting 
\begin{equation}\label{eq:pmettomet} % Only works with a division quantale, e.g. [0,\infty]
d^{*}(x,y)=2d(x,y)-d(x,x)-d(y,y)\tag{$\star$}
\end{equation}


For a more telling and somewhat archetypal example, take any set $X$ and consider the set $X^{\leq \omega}$ of all sequences of elements of $X$ indexed by an ordinal less than or equal to $\omega$. For all such sequences $s,t$, let $d(s,t) = 2^{-n} \in [0,\infty]$, where $n$ is the length of the largest common prefix to $s$ and $t$: one can check that $(X^{\leq \omega}, [0,\infty], d)$ is indeed a generalized partial metric space. In fact, if we interpret the prefixes of a sequence as pieces of partial information, then we have $d(s,s) = d(s,t)$ if and only if $t$ is a refinement of $s$ (\textit{i.e.} if it contains more information), and $d(s,s) = 0$ if and only if $s$ is total (\textit{i.e.} if it cannot be refined).

One can check that for all partial metric spaces $(X, Q, d_X)$ and $(Y, R, d_Y)$, $(X \times Y, Q \times R, d_{X \times Y})$ is a generalized partial metric space, where $d_{X \times Y}((x_1, y_1), (x_2, y_2)) = (d_X(x_1, x_2), d_Y(y_1, y_2))$. However, in general, it is not clear how one should define a partial metric on a function space. In  Section \ref{subsection:type-gpms} we introduce a construction to obtain partial metric spaces on function spaces by generalizing some properties of the standard diameter function on sets of real numbers.
