% !TEX root = CSL 2021.tex


In the previous section we showed how to associate each simple type $A$ with a  partial metric $d_{A}$ over the closed terms of type $A$. 
We  now illustrate through a few basic examples how the higher-order and metric features of this semantics can be used to formalize contextual reasoning about program differences.


To make our examples more realistic, we will consider some natural extensions of $\STLC$.
It is not difficult to see that all constructions from Section 3 still work if we add to $\STLC$ some new base types. For example, we can add to our language a type $\mathsf{Nat}$ for natural numbers, indicating for each $n\in \mathbb N$, the corresponding normal forms of $\mathsf{Nat}$ as $\mathtt n$. A natural choice is to let   
$\intervals{\mathsf{Nat}}=\{ \{ t \mid \exists n\in a \ t\leadsto \mathtt n\}\mid a \text{ finite subset of }\mathbb N \text{ or }a=\mathbb N\}$, $\distances{\mathsf{Nat}}=[0,\infty]$ and $d_{\mathsf{Nat}}(t,u)=| n-m|$, where $t\to^{*}_{\beta}\mathtt n$ and $u\to^{*}_{\beta}\mathtt m$. 

Moreover, our constructions scale well also to extensions of $\STLC$ obtained by adding new program constructors, as soon as these do not compromise the existence and uniqueness of normal forms (since the fact that closed programs of type $\mathsf{Real}$ have a normal form plays an important role to define $\intervals{\mathsf{Real}}$).  
For instance, if we suppose that all programs of type $\mathsf{Real}\to\mathsf{Real}$ in $\STLC$ are either 
differentiable or integrable (see Remark \ref{rem:continuous}), we can consider extension of $\STLC$ with  differential or integral operators, as in $\mathsf{Real\ PCF}$ \cite{Di-Gianantonio:2013aa, Edalat:2000aa}.



We start with a classical example from approximate computing that we adapt from \cite{chaudhuri}. 

\begin{example}[Loop perforation]
We work in the extension of $\STLC$ with a type $\mathsf{Nat}$.
We discuss a transformation that replaces a program $t$ which performs $n$ iterations by a program which only performs the iterations $0,k,2k,3k,\dots$, each repeated $k$ times. 

%Let $\mathsf{Nat}$ be a suitable type representing natural numbers in $\STLC$ and s
Suppose  $t: (A\times A\to A) \to \mathsf{Nat}\to (A\to A)\to A$, for $n\geq 1$, is a term such that $th\mathtt n f$ 
computes the $n$-times iteration of $h$ as follows: $th \mathtt 0f= h\langle f\mathtt 0, f\mathtt 0\rangle$ and $th(\mathtt{n+1})f=h\langle th\mathtt n f, f(\mathtt{n+1})\rangle$. 
Let $\mathsf{Perf}^{k}(t)$, the $k$-th perforation of $t$, be the program   
$(\mathsf{Perf}^{k}(t))h\mathtt nf= t(\lambda x. (h^{(k)}x)) \mathtt{\lfloor n\rfloor_{k}} (\lambda x. f(x* \mathtt k)$, where $\lfloor n\rfloor_{k}$ indicates the least $m\leq n$ such that $m$ is divisible by $k$, and $x*\mathtt k$ is the multiplication of $x$ by $k$. 





To compute the distance 
$d_{A}(v_{n},w_{n}    )$ between  $v_{n}=th\mathtt n f $ and its perforation $w_{n}=\mathsf{Perf}^{k}(t)h\mathtt nf$  we can reason as follows: 
\begin{itemize}

\item[i.] $v_{n}$ performs $n$-iterations while $w_{n}$ performs $k\lfloor n\rfloor_{k}  \leq n$ iterations, and we can compute  
$d_{A}(v_{n}, v_{(k \lfloor n\rfloor_{k})})$ as the diameter of 
$\partial(t)\partial(h)([ k \lfloor n\rfloor_{k}, n]_{\mathsf{Nat}}) \partial(f)$.




\item[ii.] If $n$ is divisible by $k$, then for $i\leq n$, at the $i$-th iteration of $v_{n}$ the function $f$ is applied  to $\mathtt i$, while at the $i$-th iteration of $w_{n}$, $f$ is applied to $\lfloor i\rfloor_{k}$. Now, the error of replacing  $f\mathtt i$ by $ f\lfloor \mathtt j\rfloor_{k}$, with $\mathtt i,\mathtt j$ in some $a\in \intervals{\mathsf{Nat}}$, is accounted for by the approximate program $c[y]= \partial(f)(y-k  )$, where $y-k= y \vee \{u-\mathtt k\mid u \in y\}$.
We deduce then that 
$d_{A}(v_{n}, w_{n})$ is bounded by the diameter of $\partial(t)\partial(h)\tointerval{\mathtt n} (\lambda y.c[y])$.

\item[iii.] From the fact that $w_{n}=w_{(k\cdot \lfloor n\rfloor_{k})}$ and the triangular inequality of the partial metric $d_{A}$ we deduce  
$d_{A}(v_{n}, w_{n})=
d_{A}(v_{n},w_{(k\cdot \lfloor n\rfloor_{k})}) \leq
d_{A}(v_{n}, v_{(k\cdot \lfloor n\rfloor_{k})})+
d_{A}(v_{(k\cdot \lfloor n\rfloor_{k})}, w_{(k\cdot \lfloor n\rfloor_{k})})-
d_{A}(v_{(k\cdot \lfloor n\rfloor_{k})},v_{(k\cdot \lfloor n\rfloor_{k})} )$


\end{itemize}

From facts i.-iii. we deduce an explicit bound for $d_{A}(v_{n},w_{n}    )$ in terms of $\partial(t), \partial(f)$ and $n$: \\
\adjustbox{scale=0.9}{
$d_{A}(v_{n},w_{n}    )\leq 
 \diam_{A}(\partial(t)\partial(h)([ k \lfloor n\rfloor_{k}, n]_{\mathsf{Nat}}) \partial(f))+
 \diam_{A}(\partial(t)\partial(h)\tointerval{\mathtt n} (\lambda y.\partial(f)(y- k))) -
 \diam_{A}(\partial(t)\partial(h)\tointerval{\mathtt n} \partial(f)) 
$.}

\end{example}

We now show how the partial metric semantics can be used to reason about 
basic approximation techniques from numerical analysis.  


\begin{example}[Taylor approximation]

We assume that all programs of type $\mathsf{Real}\to\mathsf{Real}$ in $\STLC$ are differentiable and that for all $n$, program $t:\mathsf{Real}\to\mathsf{Real}$ and real number $r$, we can define a term 
%$\mathtt T^{n}: ((\mathsf{Real}\to \mathsf{Real})\times \mathsf{Real})\to \mathsf{Real}\to \mathsf{Real}$ such that 
$\mathtt T^{n}( t, r):\mathsf{Real}\to\mathsf{Real}$ computing the $n$-th truncated Taylor polynomial of $t$ at $r$. 
%Then, given a term $t: \mathsf{Real}\to \mathsf{Real}$, 
The distance 
$d_{\mathsf{Real}\to\mathsf{Real}}(t, \mathtt T^{n}( t,\mathtt 0 ))$ is the map associating an interval $a$ with the diameter of the smallest interval containing the image of $a$ under both $t$ and $\mathtt T^{n}( t,\mathtt 0 )$. 
This value will approximately converge to the self-distance of $t$ when $a$ is a small interval of $0$, and will tend to diverge when $a$ contains points which are far enough from 0. 

%between $t$ and its Taylor expansion within a context  $\mathsf C[\ ]= [\ ] r$ which applies them to some value $r$, can be computed as a function of the interval $[0,r]$.   
For example, if $t$ is the function $t=\lambda x.\sin(x)$, and $a$ is an interval of $0$, then using standard analytic reasoning we can compute a bound
$d_{\mathsf{Real}\to \mathsf{Real}}(t, \mathtt T^{n}( t,\mathtt 0 ))(a  )\leq \frac{\diam_{\mathsf{Real}}(a)^{n+1}}{(n+1)!} $, which tends to $0$ as the diameter of $a$ tends to $0$.

Observe that if, instead, we used the $\sup$-distance $d_{\sup}(t,u)= \sup\{d_{\mathsf{Real}}(tr, ur)\mid r\in \Lambda_{\mathsf{Real}}\}$, then we could not reason as above, since  the $\sup$-distance between $\lambda x.\sin(x)$ and its  truncated Taylor polynomials is infinite.  

\end{example}

\begin{example}[Integral approximation]
We now assume that all functions in $\mathcal F_{n}$ are integrable and that we have (see \cite{Edalat:2000aa}) at our disposal a program $\lambda fx.\mathsf I_{[0,x]}(f): (\mathsf{Real}\to \mathsf{Real})\to \mathsf{Real}\to \mathsf{Real}$ such that $\mathsf I_{[0,r]}(t)$ computes (a precise enough approximation of) the definite integral $\int_{0}^{|r|}tx \ dx$.
In many contexts we might prefer to replace the expensive computation of $\mathsf I_{[0,r]}(t)$ by the (more economical but less precise) computation of a finite Riemann sum $\mathsf R^{n}_{[0,r]}(t)=  \sum_{i=1}^{n}(tx_{i})\cdot |r|/n$, where 
 $x_{i}=  i\cdot |r|/n$.  


Suppose now that, in order to approximate the integral of some computationally expensive program $t$ on $[0,r]$, we replace $t$ by some more efficient program $u$ which, over $[0,r]$, is very close to $t$. Let $\varepsilon_{t}(r)$ indicate the distance between the true integral of $t$ over $[0,r]$ and $\mathsf{R}^{n}_{[0,r]}(t)$,
 and moreover let 
$\eta_{t,u}(r)$ be the diameter of $\partial(t)([0,r])\vee\partial(u)([0,r])$.

Using the metric structure of $\mathsf{Real}$ we can then bound the error we incur in by replacing the true integral \emph{of $t$} with the Riemann sum \emph{of $u$}. 
In fact, by standard calculation we can compute the bound
$d_{\mathsf{Real}} ( \mathsf{R}^{n}_{[0,r]}(t), \mathsf R^{n}_{[0,r]}(u))\leq 
d_{\mathsf{Real}\to\mathsf{Real}}(t, u)([0,r]) \cdot |r|=\eta_{t,u}(r) \cdot |r|$.
Then, using the triangular inequality of the standard metric on $\mathsf{Real}$ we deduce
\begin{align*}
 d_{\mathsf{Real}} ( \mathsf{I}_{[0,r]}(t), \mathsf R^{n}_{[0,r]}(u)) \leq
d_{\mathsf{Real}} ( \mathsf{I}_{[0,r]}(t), \mathsf R^{n}_{[0,r]}(t))  & +
d_{\mathsf{Real}} ( \mathsf{R}_{[0,r]}(t),\mathsf R^{n}_{[0,r]}(u)) \\
& \leq \varepsilon_{t}(r)
+ 
\eta_{t,u}(r)\cdot 
  |r|
 \end{align*}
 Using the partial metric on $\mathsf{Real}\to \mathsf{Real}$, we can also derive a bound expressing how much the error above is \emph{sensitive to changes of $r$}. 
First, using standard analytic techniques (under suitable assumptions for $t$ and its derivatives) one can find a program $v:\mathsf{Real}\to\mathsf{Real}$ such that $vr $ computes an upper bound for $\varepsilon_{t}(r)$. 
Then, using the triangular inequality of the partial metric on $\mathsf{Real}\to \mathsf{Real}$ we deduce, for all interval $a$, the following bound:
\begin{align*}
& d_{\mathsf{Real}\to\mathsf{Real}} (\lambda x. \mathsf{I}_{[0,x]}(t),\lambda x. \mathsf R^{n}_{[0,x]}(u))(a) \\
 &\leq \
d_{\mathsf{Real}\to\mathsf{Real}} ( \lambda x.\mathsf{I}_{[0,x]}(t), \lambda x.\mathsf R^{n}_{[0,x]}(t))(a) +
d_{\mathsf{Real}\to\mathsf{Real}} ( \lambda x.\mathsf{R}_{[0,x]}(t), \lambda x.\mathsf R^{n}_{[0,x]}(u))(a) \\
 & \qquad\qquad\qquad\qquad\qquad\qquad\qquad\qquad\qquad\qquad
- d_{\mathsf{Real}\to\mathsf{Real}} ( \lambda x.\mathsf{R}_{[0,x]}(t), \allowbreak \lambda x. \mathsf R^{n}_{0,x]}(t))(a) \\
 & \leq \ 
d_{\mathsf{Real}\to \mathsf{Real}}(v,v)(a)
 +
 \big( d_{\mathsf{Real}\to\mathsf{Real}}(t,u)(a)- d_{\mathsf{Real}\to\mathsf{Real}}(t,t)(a)\big)\cdot \diam_{\mathsf{Real}}(a)
\end{align*}
\end{example}





