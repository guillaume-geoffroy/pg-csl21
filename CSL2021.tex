
\documentclass[a4paper,USenglish,cleveref, autoref, thm-restate]{lipics-v2019}
%This is a template for producing LIPIcs articles. 
%See lipics-manual.pdf for further information.
%for A4 paper format use option "a4paper", for US-letter use option "letterpaper"
%for british hyphenation rules use option "UKenglish", for american hyphenation rules use option "USenglish"
%for section-numbered lemmas etc., use "numberwithinsect"
%for enabling cleveref support, use "cleveref"
%for enabling autoref support, use "autoref"
%for anonymousing the authors (e.g. for double-blind review), add "anonymous"
%for enabling thm-restate support, use "thm-restate"

%\graphicspath{{./graphics/}}%helpful if your graphic files are in another directory

\bibliographystyle{plainurl}% the mandatory bibstyle
 
\title{A Partial Metric Semantics of Higher-Order Types and Approximate Program Transformations}

\titlerunning{A Partial Metric Semantics of Higher-Order Programs} % optional, please use if title is longer than one line

\author{Guillaume Geoffroy}
{Universit\`a di Bologna, Dipartimento Informatica, Scienza e Ingegneria, Italy}
{guillaume.geoffroy@unibo.it} 
{}
{ERC CoG 818616 “DIAPASoN”, ANR 16CE250011 “REPAS”}

\author{Paolo Pistone}
{Universit\`a di Bologna, Dipartimento Informatica, Scienza e Ingegneria, Italy}
{paolo.pistone2@unibo.it} 
{}
{ERC CoG 818616 “DIAPASoN”, ANR 16CE250011 “REPAS”}
%{johnqpublic@dummyuni.org}{https://orcid.org/0000-0002-1825-0097}{(Optional) author-specific funding acknowledgements}%TODO mandatory, please use full name; only 1 author per \author macro; first two parameters are mandatory, other parameters can be empty. Please provide at least the name of the affiliation and the country. The full address is optional


\authorrunning{G. Geoffroy and P. Pistone} % mandatory. First: Use abbreviated first/middle names. Second (only in severe cases): Use first author plus 'et al.'

\Copyright{G. Geoffroy and P. Pistone} % mandatory, please use full first names. LIPIcs license is "CC-BY";  http://creativecommons.org/licenses/by/3.0/

\ccsdesc[500]{Theory of computation~Denotational semantics} % Please choose ACM 2012 classifications from https://dl.acm.org/ccs/ccs_flat.cfm

\keywords{Simply typed $\lambda$-calculus, program metrics, approximate program transformations, partial metric spaces}  % mandatory; please add comma-separated list of keywords

%\category{} %optional, e.g. invited paper

\relatedversion{} %optional, e.g. full version hosted on arXiv, HAL, or other respository/website
%\relatedversion{A full version of the paper is available at \url{...}.}

\supplement{}%optional, e.g. related research data, source code, ... hosted on a repository like zenodo, figshare, GitHub, ...

%\funding{(Optional) general funding statement \dots}%optional, to capture a funding statement, which applies to all authors. Please enter author specific funding statements as fifth argument of the \author macro.

%\acknowledgements{I want to thank \dots}%optional

%\nolinenumbers %uncomment to disable line numbering

%\hideLIPIcs  %uncomment to remove references to LIPIcs series (logo, DOI, ...), e.g. when preparing a pre-final version to be uploaded to arXiv or another public repository

%Editor-only macros:: begin (do not touch as author)%%%%%%%%%%%%%%%%%%%%%%%%%%%%%%%%%%
\EventEditors{Christel Baier and Jean Goubault-Larrecq}
\EventNoEds{2}
\EventLongTitle{29th EACSL Annual Conference on Computer Science Logic (CSL 2021)}
\EventShortTitle{CSL 2021}
\EventAcronym{CSL}
\EventYear{2021}
\EventDate{January 25--28, 2021}
\EventLocation{Ljubljana, Slovenia (Virtual Conference)}
\EventLogo{}
\SeriesVolume{183}
\ArticleNo{32}
%%%%%%%%%%%%%%%%%%%%%%%%%%%%%%%%%%%%%%%%%%%%%%%%%%%%%%


%packages


\usepackage{booktabs}   %% For formal tables:
                        %% http://ctan.org/pkg/booktabs
\usepackage{subcaption} %% For complex figures with subfigures/subcaptions
%% http://ctan.org/pkg/subcaption
\usepackage{amsmath, amssymb}
\usepackage{xifthen}
\usepackage{tikz,pgfplots}

\usepackage{stmaryrd}
\usetikzlibrary{cd}

\usepackage{rotating}
\usepackage{comment}
\usepackage{adjustbox}
\usepackage{pgfplots}
\pgfplotsset{compat=1.9}
\usepackage{caption}

\usetikzlibrary{intersections}



\tikzset{commutative diagrams/.cd,
mysymbol/.style={start anchor=center,end anchor=center,draw=none}
}
\newcommand\Lessone[2][\begin{rotate}{25}{\huge$\leq$}\end{rotate}]{%
  \arrow[mysymbol]{#2}[description]{#1}}
\newcommand\Lesstwo[2][\begin{rotate}{25}{\huge$\geq$}\end{rotate}]{%
  \arrow[mysymbol]{#2}[description]{#1}}



%macros


%% Custom command definitions
\newcommand{\ob}[1]{\operatorname{Ob}\left(#1\right)}
\newcommand{\close}[1]{\overline{#1}}
\newcommand{\I}{\mathcal{I}}
\newcommand{\J}{\mathcal{J}}
\newcommand{\K}{\mathcal{K}}
\newcommand{\apcat}{\mathcal{A}}
\newcommand{\apcatT}{\mathcal{A}^{\tang}}
\newcommand{\apcatd}{\mathcal{A}_d}
\newcommand{\pmcat}{\mathcal{M}}
\newcommand{\epcat}{\mathcal{E}}
\newcommand{\diff}{\partial}
\newcommand{\tang}{T}
\newcommand{\pmofap}[1]{\ifthenelse{\isempty{#1}}{\left\vert-\right\vert}{\left\vert#1\right\vert}}
\newcommand{\setcat}{\operatorname{Set}}
\newcommand{\posetcat}{\operatorname{Poset}}
\newcommand{\R}{\mathbb{R}}
\newcommand{\cart}{\times}%{\mathbin{\&}}
\newcommand{\projL}{\pi_1}
\newcommand{\projR}{\pi_2}

\newcommand{\oneap}{1}
\newcommand{\onequantale}{1}

\newcommand{\id}[1][]{\ifthenelse{\isempty{#1}}{\operatorname{id}}{\operatorname{id}_{#1}}}

\newcommand{\ev}{\operatorname{ev}}
\newcommand{\lam}{\lambda}

\newcommand{\dsp}{\operatorname{Diam}}
\newcommand{\bs}[1]{\left\vert#1\right\vert}
\newcommand{\set}{\operatorname{Set}}
\newcommand{\eff}{\operatorname{Eff}}
\newcommand{\scott}{\operatorname{Scott}}

\newcommand{\STLC}{\mathsf{ST\lambda C}(\mathcal F_{n})}
\newcommand{\QMet}{\mathsf{Met}_Q}
\newcommand{\RMet}{\mathsf{Met}_{\R_+^\infty}}
\newcommand{\gpms}[1]{#1}
\newcommand{\real}{\R}
\newcommand{\To}[2]{#1 \Rightarrow #2}
\newcommand{\abs}{\lambda}


\newcommand{\ball}{\mathsf{B}}


\newcommand{\B}[1]{\mathbf{#1}}%{\mathbin{\&}}
\newcommand{\powerset}{\mathcal{P}}

\newcommand{\intervals}[1]{\llbracket #1 \rrbracket}
\newcommand{\distances}[1]{\llparenthesis #1 \rrparenthesis}
\newcommand{\tointerval}[1]{\overline{#1}}
\newcommand{\quantaleleq}{\geq}
\newcommand{\quantalegeq}{\leq}
\newcommand{\quantaleop}{+}
\newcommand{\diam}{\delta}
\newcommand{\oeq}{\approx}

\newcommand{\metalambda}{%
  \mathop{%
    \rlap{$\lambda$}%
    \mkern2mu
    \raisebox{.275ex}{$\lambda$}%
  }%
}



\begin{document}

\maketitle

\begin{abstract}
Program semantics is traditionally concerned with program equivalence. However, in fields like approximate, incremental and probabilistic computation, it is often useful to describe \emph{to which extent} two programs behave in a similar, although non equivalent way. This has motivated the study of program (pseudo)metrics, which have found widespread applications, \emph{e.g.}~in differential privacy. 
In this paper we show that the standard metric on real numbers can be lifted to higher-order types in a novel way, yielding a metric semantics of the simply typed lambda-calculus in which types are interpreted as quantale-valued partial metric spaces. Using such metrics we define a class of higher-order denotational models, called diameter space models, that provide a quantitative semantics of approximate program transformations.
Noticeably, the distances between objects of higher-types are elements of functional, thus non-numerical, quantales. This allows us to model contextual reasoning about arbitrary functions, thus deviating from classic metric semantics.
\end{abstract}

\section{Introduction}

\input{section-introduction}


\section{Generalized Partial Metric Spaces}

\input{section-preliminaries}

\section{Approximate Programs for the Simply-Typed $\lambda$-Calculus over $\mathsf{Real}$}
\label{section:stlc}

\input{section-stlc}

\section{Computing Program Distances using Partial Metrics}

% !TEX root = CSL 2021.tex


In the previous section we showed how to associate each simple type $A$ with a  partial metric $d_{A}$ over the closed terms of type $A$. 
We  now illustrate through a few basic examples how the higher-order and metric features of this semantics can be used to formalize contextual reasoning about program differences.


To make our examples more realistic, we will consider some natural extensions of $\STLC$.
It is not difficult to see that all constructions from Section 3 still work if we add to $\STLC$ some new base types. For example, we can add to our language a type $\mathsf{Nat}$ for natural numbers, indicating for each $n\in \mathbb N$, the corresponding normal forms of $\mathsf{Nat}$ as $\mathtt n$. A natural choice is to let   
$\intervals{\mathsf{Nat}}=\{ \{ t \mid \exists n\in a \ t\leadsto \mathtt n\}\mid a \text{ finite subset of }\mathbb N \text{ or }a=\mathbb N\}$, $\distances{\mathsf{Nat}}=[0,\infty]$ and $d_{\mathsf{Nat}}(t,u)=| n-m|$, where $t\to^{*}_{\beta}\mathtt n$ and $u\to^{*}_{\beta}\mathtt m$. 

Moreover, our constructions scale well also to extensions of $\STLC$ obtained by adding new program constructors, as soon as these do not compromise the existence and uniqueness of normal forms (since the fact that closed programs of type $\mathsf{Real}$ have a normal form plays an important role to define $\intervals{\mathsf{Real}}$).  
For instance, if we suppose that all programs of type $\mathsf{Real}\to\mathsf{Real}$ in $\STLC$ are either 
differentiable or integrable (see Remark \ref{rem:continuous}), we can consider extension of $\STLC$ with  differential or integral operators, as in $\mathsf{Real\ PCF}$ \cite{Di-Gianantonio:2013aa, Edalat:2000aa}.



We start with a classical example from approximate computing that we adapt from \cite{chaudhuri}. 

\begin{example}[Loop perforation]
We work in the extension of $\STLC$ with a type $\mathsf{Nat}$.
We discuss a transformation that replaces a program $t$ which performs $n$ iterations by a program which only performs the iterations $0,k,2k,3k,\dots$, each repeated $k$ times. 

%Let $\mathsf{Nat}$ be a suitable type representing natural numbers in $\STLC$ and s
Suppose  $t: (A\times A\to A) \to \mathsf{Nat}\to (A\to A)\to A$, for $n\geq 1$, is a term such that $th\mathtt n f$ 
computes the $n$-times iteration of $h$ as follows: $th \mathtt 0f= h\langle f\mathtt 0, f\mathtt 0\rangle$ and $th(\mathtt{n+1})f=h\langle th\mathtt n f, f(\mathtt{n+1})\rangle$. 
Let $\mathsf{Perf}^{k}(t)$, the $k$-th perforation of $t$, be the program   
$(\mathsf{Perf}^{k}(t))h\mathtt nf= t(\lambda x. (h^{(k)}x)) \mathtt{\lfloor n\rfloor_{k}} (\lambda x. f(x* \mathtt k)$, where $\lfloor n\rfloor_{k}$ indicates the least $m\leq n$ such that $m$ is divisible by $k$, and $x*\mathtt k$ is the multiplication of $x$ by $k$. 





To compute the distance 
$d_{A}(v_{n},w_{n}    )$ between  $v_{n}=th\mathtt n f $ and its perforation $w_{n}=\mathsf{Perf}^{k}(t)h\mathtt nf$  we can reason as follows: 
\begin{itemize}

\item[i.] $v_{n}$ performs $n$-iterations while $w_{n}$ performs $k\lfloor n\rfloor_{k}  \leq n$ iterations, and we can compute  
$d_{A}(v_{n}, v_{(k \lfloor n\rfloor_{k})})$ as the diameter of 
$\partial(t)\partial(h)([ k \lfloor n\rfloor_{k}, n]_{\mathsf{Nat}}) \partial(f)$.




\item[ii.] If $n$ is divisible by $k$, then for $i\leq n$, at the $i$-th iteration of $v_{n}$ the function $f$ is applied  to $\mathtt i$, while at the $i$-th iteration of $w_{n}$, $f$ is applied to $\lfloor i\rfloor_{k}$. Now, the error of replacing  $f\mathtt i$ by $ f\lfloor \mathtt j\rfloor_{k}$, with $\mathtt i,\mathtt j$ in some $a\in \intervals{\mathsf{Nat}}$, is accounted for by the approximate program $c[y]= \partial(f)(y-k  )$, where $y-k= y \vee \{u-\mathtt k\mid u \in y\}$.
We deduce then that 
$d_{A}(v_{n}, w_{n})$ is bounded by the diameter of $\partial(t)\partial(h)\tointerval{\mathtt n} (\lambda y.c[y])$.

\item[iii.] From the fact that $w_{n}=w_{(k\cdot \lfloor n\rfloor_{k})}$ and the triangular inequality of the partial metric $d_{A}$ we deduce  
$d_{A}(v_{n}, w_{n})=
d_{A}(v_{n},w_{(k\cdot \lfloor n\rfloor_{k})}) \leq
d_{A}(v_{n}, v_{(k\cdot \lfloor n\rfloor_{k})})+
d_{A}(v_{(k\cdot \lfloor n\rfloor_{k})}, w_{(k\cdot \lfloor n\rfloor_{k})})-
d_{A}(v_{(k\cdot \lfloor n\rfloor_{k})},v_{(k\cdot \lfloor n\rfloor_{k})} )$


\end{itemize}

From facts i.-iii. we deduce an explicit bound for $d_{A}(v_{n},w_{n}    )$ in terms of $\partial(t), \partial(f)$ and $n$: \\
\adjustbox{scale=0.9}{
$d_{A}(v_{n},w_{n}    )\leq 
 \diam_{A}(\partial(t)\partial(h)([ k \lfloor n\rfloor_{k}, n]_{\mathsf{Nat}}) \partial(f))+
 \diam_{A}(\partial(t)\partial(h)\tointerval{\mathtt n} (\lambda y.\partial(f)(y- k))) -
 \diam_{A}(\partial(t)\partial(h)\tointerval{\mathtt n} \partial(f)) 
$.}

\end{example}

We now show how the partial metric semantics can be used to reason about 
basic approximation techniques from numerical analysis.  


\begin{example}[Taylor approximation]

We assume that all programs of type $\mathsf{Real}\to\mathsf{Real}$ in $\STLC$ are differentiable and that for all $n$, program $t:\mathsf{Real}\to\mathsf{Real}$ and real number $r$, we can define a term 
%$\mathtt T^{n}: ((\mathsf{Real}\to \mathsf{Real})\times \mathsf{Real})\to \mathsf{Real}\to \mathsf{Real}$ such that 
$\mathtt T^{n}( t, r):\mathsf{Real}\to\mathsf{Real}$ computing the $n$-th truncated Taylor polynomial of $t$ at $r$. 
%Then, given a term $t: \mathsf{Real}\to \mathsf{Real}$, 
The distance 
$d_{\mathsf{Real}\to\mathsf{Real}}(t, \mathtt T^{n}( t,\mathtt 0 ))$ is the map associating an interval $a$ with the diameter of the smallest interval containing the image of $a$ under both $t$ and $\mathtt T^{n}( t,\mathtt 0 )$. 
This value will approximately converge to the self-distance of $t$ when $a$ is a small interval of $0$, and will tend to diverge when $a$ contains points which are far enough from 0. 

%between $t$ and its Taylor expansion within a context  $\mathsf C[\ ]= [\ ] r$ which applies them to some value $r$, can be computed as a function of the interval $[0,r]$.   
For example, if $t$ is the function $t=\lambda x.\sin(x)$, and $a$ is an interval of $0$, then using standard analytic reasoning we can compute a bound
$d_{\mathsf{Real}\to \mathsf{Real}}(t, \mathtt T^{n}( t,\mathtt 0 ))(a  )\leq \frac{\diam_{\mathsf{Real}}(a)^{n+1}}{(n+1)!} $, which tends to $0$ as the diameter of $a$ tends to $0$.

Observe that if, instead, we used the $\sup$-distance $d_{\sup}(t,u)= \sup\{d_{\mathsf{Real}}(tr, ur)\mid r\in \Lambda_{\mathsf{Real}}\}$, then we could not reason as above, since  the $\sup$-distance between $\lambda x.\sin(x)$ and its  truncated Taylor polynomials is infinite.  

\end{example}

\begin{example}[Integral approximation]
We now assume that all functions in $\mathcal F_{n}$ are integrable and that we have (see \cite{Edalat:2000aa}) at our disposal a program $\lambda fx.\mathsf I_{[0,x]}(f): (\mathsf{Real}\to \mathsf{Real})\to \mathsf{Real}\to \mathsf{Real}$ such that $\mathsf I_{[0,r]}(t)$ computes (a precise enough approximation of) the definite integral $\int_{0}^{|r|}tx \ dx$.
In many contexts we might prefer to replace the expensive computation of $\mathsf I_{[0,r]}(t)$ by the (more economical but less precise) computation of a finite Riemann sum $\mathsf R^{n}_{[0,r]}(t)=  \sum_{i=1}^{n}(tx_{i})\cdot |r|/n$, where 
 $x_{i}=  i\cdot |r|/n$.  


Suppose now that, in order to approximate the integral of some computationally expensive program $t$ on $[0,r]$, we replace $t$ by some more efficient program $u$ which, over $[0,r]$, is very close to $t$. Let $\varepsilon_{t}(r)$ indicate the distance between the true integral of $t$ over $[0,r]$ and $\mathsf{R}^{n}_{[0,r]}(t)$,
 and moreover let 
$\eta_{t,u}(r)$ be the diameter of $\partial(t)([0,r])\vee\partial(u)([0,r])$.

Using the metric structure of $\mathsf{Real}$ we can then bound the error we incur in by replacing the true integral \emph{of $t$} with the Riemann sum \emph{of $u$}. 
In fact, by standard calculation we can compute the bound
$d_{\mathsf{Real}} ( \mathsf{R}^{n}_{[0,r]}(t), \mathsf R^{n}_{[0,r]}(u))\leq 
d_{\mathsf{Real}\to\mathsf{Real}}(t, u)([0,r]) \cdot |r|=\eta_{t,u}(r) \cdot |r|$.
Then, using the triangular inequality of the standard metric on $\mathsf{Real}$ we deduce
\begin{align*}
 d_{\mathsf{Real}} ( \mathsf{I}_{[0,r]}(t), \mathsf R^{n}_{[0,r]}(u)) \leq
d_{\mathsf{Real}} ( \mathsf{I}_{[0,r]}(t), \mathsf R^{n}_{[0,r]}(t))  & +
d_{\mathsf{Real}} ( \mathsf{R}_{[0,r]}(t),\mathsf R^{n}_{[0,r]}(u)) \\
& \leq \varepsilon_{t}(r)
+ 
\eta_{t,u}(r)\cdot 
  |r|
 \end{align*}
 Using the partial metric on $\mathsf{Real}\to \mathsf{Real}$, we can also derive a bound expressing how much the error above is \emph{sensitive to changes of $r$}. 
First, using standard analytic techniques (under suitable assumptions for $t$ and its derivatives) one can find a program $v:\mathsf{Real}\to\mathsf{Real}$ such that $vr $ computes an upper bound for $\varepsilon_{t}(r)$. 
Then, using the triangular inequality of the partial metric on $\mathsf{Real}\to \mathsf{Real}$ we deduce, for all interval $a$, the following bound:
\begin{align*}
& d_{\mathsf{Real}\to\mathsf{Real}} (\lambda x. \mathsf{I}_{[0,x]}(t),\lambda x. \mathsf R^{n}_{[0,x]}(u))(a) \\
 &\leq \
d_{\mathsf{Real}\to\mathsf{Real}} ( \lambda x.\mathsf{I}_{[0,x]}(t), \lambda x.\mathsf R^{n}_{[0,x]}(t))(a) +
d_{\mathsf{Real}\to\mathsf{Real}} ( \lambda x.\mathsf{R}_{[0,x]}(t), \lambda x.\mathsf R^{n}_{[0,x]}(u))(a) \\
 & \qquad\qquad\qquad\qquad\qquad\qquad\qquad\qquad\qquad\qquad
- d_{\mathsf{Real}\to\mathsf{Real}} ( \lambda x.\mathsf{R}_{[0,x]}(t), \allowbreak \lambda x. \mathsf R^{n}_{0,x]}(t))(a) \\
 & \leq \ 
d_{\mathsf{Real}\to \mathsf{Real}}(v,v)(a)
 +
 \big( d_{\mathsf{Real}\to\mathsf{Real}}(t,u)(a)- d_{\mathsf{Real}\to\mathsf{Real}}(t,t)(a)\big)\cdot \diam_{\mathsf{Real}}(a)
\end{align*}
\end{example}







\section{Diameter Space Models Over a Cartesian Closed Category}

\input{section-categorical-models}

\section{Conclusions}

% !TEX root = CSL 2021.tex

 
\subparagraph*{Related Work}

As stated in the introduction, differential logical relations \cite{dallago:differential-stlc} are a primary source of inspiration for our approach.
A related, but more syntactic approach to approximate program transformations is that of Westbrook and Chauduri \cite{chaudhuri}, who use a System F-based type system with a type of real numbers and an explicit distinction between exact and approximate programs.
%, and provides typing rules to formalize contextual reasoning about program differences. 
%The latter are taken as functions relating errors in input and errors in output, but the viewpoint of program metric is not considered.
Most examples of contextual reasoning from \cite{chaudhuri} can be  reformulated in our framework (as the case of loop perforation discussed in Section 4). 



The literature on program pseudo-metrics is vast. A major distinction can be made between those approaches in which metrics account for \emph{extensional} aspects of programs (like ours), 
 and approaches in which metrics are used to characterize more \emph{intensional} aspects.
To the first family belong all metric models developed for reasoning about differential privacy \cite{10.1145/1932681.1863568, 10.1007/978-3-642-29420-4_3, Barthe_2012},  
probabilistic computation \cite{10.1109/LICS.2015.64, 10.1007/978-3-662-54434-1_13} and co-inductive models \cite{DESHARNAIS2004323, VANBREUGEL2005115, 10.1007/978-3-662-44584-6_4,10.1007/3-540-48224-5_35}.
%In particular, several approaches like \cite{} emphasize the importance to formalize contextual reasoning, wich can be assured in frameworks like \cite{} by the restriction to Lipschitz-continuous or non-expansive functions. 
To the second class belong approaches like \cite{Escardo1999} which recovers the Scott model of PCF through a ultrametric semantics, and most models based on partial metric spaces \cite{Bukatin1997,doi:10.1111/j.1749-6632.1994.tb44144.x}, which rely on a correspondence between continuous Scott domains and the $T_{0}$ topology of partial metrics.

From a more mathematical viewpoint, 
\cite{Stubbe2009} discusses a characterization of exponentiable GPMS, showing that no such category can both be cartesian closed and contain the standard metric on $\mathbb R$. This result seems to add further evidence of the necessity of considering metrics over varying quantales in order to model higher-order languages. 
Finally, the elegant categorical approach to GPMS based on \emph{quantaloid-enriched categories} from \cite{Stubbe2018} seems to provide the relevant structure to develop explicit typing rules for our approximate programs.

%
% approach might provide good 
%categorical insights to improve our present understnd of program distances.

%
%. 
%The central motivation such approaches is the observation that a partial metric $d$ on a set $X$ induces an order relation given by $x\preceq_{d}y$ iff $d(x,y)\leq d(x,x)$, turning $X$ into a continuous Scott domain and, conversely, any continuous Scott domain with a countable basis is induced by a partial metric in this way.
%This allows to reformulate several classical results on denotational semantics using the $T_{1}$ topology of partial metric spaces.
%


%
%While in all these results distances are computed over a \emph{fixed} quantale, the generalization of this categorical approach to varying quantales seems to be a yet unexplored research direction.






\subparagraph*{Future Work}


%
%
%In this paper we constructed a (non-extensional) model of the simply typed $\lambda$-calculus based on generalized partial metric spaces. Our model provides a 
% differential semantics of higher-order programs, that is, a semantic description of \emph{differences} between
% higher-order programs. 
% The main novelty is that we take as morphisms between metric spaces approximate functions, \emph{i.e.} monotone functions over intervals, rather than continuous functions of some kind. While approximate functions represent sets of similar programs, usual, exact, programs can be embedded in the model through a differentiation operator. 
% This approach allows us to overcome the well-known obstacle that usual categories of metric spaces and continuous functions are not cartesian closed, and therefore cannot be models of $\STLC$.
%% 
% Instead, we obtain 
%based on generalized partial metric spaces and
%
%Moreover, our model refines previous notions of program distances based on differential logical relations.
%
%
%the use of partial metric spaces, a well-investigated metric structure to which most fundamental properties and results on standard metric spaces scale well, 
%
%
%
%
%. More importantly, we take, as morphisms between them, approximate functions, that is, function over closed intervals, rather than (Lipschitz-) continuous or non-expansive functions over points, as in usual categories of metric spaces.

The approach we presented lends itself to further extensions and generalizations.
First, we would like to investigate the interpretation of more type constructions than those of $\STLC$ (\textit{e.g.} coproducts, recursive types, effects). Moreover, we would like to explore the possibility of exploiting the structure of the category $\dsp(\mathbb C)$ to construct new and more refined notions of approximations.
For example (we work in $\dsp(\set)$ for simplicity), 
starting from the ``standard'' set of approximate values $\mathcal I$ on $\mathbb{R}^{X\times X}$ (with elements of $\mathcal I$ being  families of compact intervals $U_{x,x'}\subseteq \mathbb R$ indexed by elements of $X$ and $X'$), one can define a new family  $\Delta^{*}\mathcal I$  of approximate values for  $\mathbb R^{X}$ by ``pulling back'' the exact map 
$\Delta:
\mathbb R^{X} \to \mathbb R^{X\times X}$ defined by $\Delta f(x,x')=f(x')-f(x)$, \textit{i.e.} letting $\Delta^{*}\mathcal I = \{ \Delta^{-1}(a) \mid a \in \mathcal I\}$. 
The new approximate values then correspond to sets of functions $f\in \mathbb R^{X}$ with a controlled variation, that is, such that $f(x')-f(x)$ is bounded by some family of intervals $U_{x,x'} \in \mathcal I$.





%
%First, in this paper we restricted our attention to $\STLC$, which is not a universal language. 
%The accommodation of full recursion in metric models is usually obtained by an application of the Banach fixed point theorem \cite{VANBREUGEL20011}. As this theorem scales well to partial metric spaces \cite{Samet:2013aa}, a natural question is whether our model, or some variant of it, can be used to provide a metric account of universal computation over the real numbers.
%


Another interesting research direction concerns probabilistic extensions of $\STLC$. 
Probabilistic metrics \cite{1029849,KOZEN1981328, 10.1109/LICS.2015.64, 10.1007/978-3-662-54434-1_13} have been the object of much research in recent years, due to the relevance of metric reasoning in some areas of computer science in which probabilistic computation plays a key role (\textit{e.g.} in cryptography \cite{GOLDWASSER1984270} and machine learning \cite{krause08robust}).
A convenient starting point seems to be the recent generalization of {probabilistic (generalized) metric spaces} to the partial metric case \cite{HE201999}.






\bibliography{CSL2021.bib}

%
%\appendix
%

\end{document}
